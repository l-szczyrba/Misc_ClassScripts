
% Default to the notebook output style

    


% Inherit from the specified cell style.




    
\documentclass[11pt]{article}

    
    
    \usepackage[T1]{fontenc}
    % Nicer default font (+ math font) than Computer Modern for most use cases
    \usepackage{mathpazo}

    % Basic figure setup, for now with no caption control since it's done
    % automatically by Pandoc (which extracts ![](path) syntax from Markdown).
    \usepackage{graphicx}
    % We will generate all images so they have a width \maxwidth. This means
    % that they will get their normal width if they fit onto the page, but
    % are scaled down if they would overflow the margins.
    \makeatletter
    \def\maxwidth{\ifdim\Gin@nat@width>\linewidth\linewidth
    \else\Gin@nat@width\fi}
    \makeatother
    \let\Oldincludegraphics\includegraphics
    % Set max figure width to be 80% of text width, for now hardcoded.
    \renewcommand{\includegraphics}[1]{\Oldincludegraphics[width=.8\maxwidth]{#1}}
    % Ensure that by default, figures have no caption (until we provide a
    % proper Figure object with a Caption API and a way to capture that
    % in the conversion process - todo).
    \usepackage{caption}
    \DeclareCaptionLabelFormat{nolabel}{}
    \captionsetup{labelformat=nolabel}

    \usepackage{adjustbox} % Used to constrain images to a maximum size 
    \usepackage{xcolor} % Allow colors to be defined
    \usepackage{enumerate} % Needed for markdown enumerations to work
    \usepackage{geometry} % Used to adjust the document margins
    \usepackage{amsmath} % Equations
    \usepackage{amssymb} % Equations
    \usepackage{textcomp} % defines textquotesingle
    % Hack from http://tex.stackexchange.com/a/47451/13684:
    \AtBeginDocument{%
        \def\PYZsq{\textquotesingle}% Upright quotes in Pygmentized code
    }
    \usepackage{upquote} % Upright quotes for verbatim code
    \usepackage{eurosym} % defines \euro
    \usepackage[mathletters]{ucs} % Extended unicode (utf-8) support
    \usepackage[utf8x]{inputenc} % Allow utf-8 characters in the tex document
    \usepackage{fancyvrb} % verbatim replacement that allows latex
    \usepackage{grffile} % extends the file name processing of package graphics 
                         % to support a larger range 
    % The hyperref package gives us a pdf with properly built
    % internal navigation ('pdf bookmarks' for the table of contents,
    % internal cross-reference links, web links for URLs, etc.)
    \usepackage{hyperref}
    \usepackage{longtable} % longtable support required by pandoc >1.10
    \usepackage{booktabs}  % table support for pandoc > 1.12.2
    \usepackage[inline]{enumitem} % IRkernel/repr support (it uses the enumerate* environment)
    \usepackage[normalem]{ulem} % ulem is needed to support strikethroughs (\sout)
                                % normalem makes italics be italics, not underlines
    

    
    
    % Colors for the hyperref package
    \definecolor{urlcolor}{rgb}{0,.145,.698}
    \definecolor{linkcolor}{rgb}{.71,0.21,0.01}
    \definecolor{citecolor}{rgb}{.12,.54,.11}

    % ANSI colors
    \definecolor{ansi-black}{HTML}{3E424D}
    \definecolor{ansi-black-intense}{HTML}{282C36}
    \definecolor{ansi-red}{HTML}{E75C58}
    \definecolor{ansi-red-intense}{HTML}{B22B31}
    \definecolor{ansi-green}{HTML}{00A250}
    \definecolor{ansi-green-intense}{HTML}{007427}
    \definecolor{ansi-yellow}{HTML}{DDB62B}
    \definecolor{ansi-yellow-intense}{HTML}{B27D12}
    \definecolor{ansi-blue}{HTML}{208FFB}
    \definecolor{ansi-blue-intense}{HTML}{0065CA}
    \definecolor{ansi-magenta}{HTML}{D160C4}
    \definecolor{ansi-magenta-intense}{HTML}{A03196}
    \definecolor{ansi-cyan}{HTML}{60C6C8}
    \definecolor{ansi-cyan-intense}{HTML}{258F8F}
    \definecolor{ansi-white}{HTML}{C5C1B4}
    \definecolor{ansi-white-intense}{HTML}{A1A6B2}

    % commands and environments needed by pandoc snippets
    % extracted from the output of `pandoc -s`
    \providecommand{\tightlist}{%
      \setlength{\itemsep}{0pt}\setlength{\parskip}{0pt}}
    \DefineVerbatimEnvironment{Highlighting}{Verbatim}{commandchars=\\\{\}}
    % Add ',fontsize=\small' for more characters per line
    \newenvironment{Shaded}{}{}
    \newcommand{\KeywordTok}[1]{\textcolor[rgb]{0.00,0.44,0.13}{\textbf{{#1}}}}
    \newcommand{\DataTypeTok}[1]{\textcolor[rgb]{0.56,0.13,0.00}{{#1}}}
    \newcommand{\DecValTok}[1]{\textcolor[rgb]{0.25,0.63,0.44}{{#1}}}
    \newcommand{\BaseNTok}[1]{\textcolor[rgb]{0.25,0.63,0.44}{{#1}}}
    \newcommand{\FloatTok}[1]{\textcolor[rgb]{0.25,0.63,0.44}{{#1}}}
    \newcommand{\CharTok}[1]{\textcolor[rgb]{0.25,0.44,0.63}{{#1}}}
    \newcommand{\StringTok}[1]{\textcolor[rgb]{0.25,0.44,0.63}{{#1}}}
    \newcommand{\CommentTok}[1]{\textcolor[rgb]{0.38,0.63,0.69}{\textit{{#1}}}}
    \newcommand{\OtherTok}[1]{\textcolor[rgb]{0.00,0.44,0.13}{{#1}}}
    \newcommand{\AlertTok}[1]{\textcolor[rgb]{1.00,0.00,0.00}{\textbf{{#1}}}}
    \newcommand{\FunctionTok}[1]{\textcolor[rgb]{0.02,0.16,0.49}{{#1}}}
    \newcommand{\RegionMarkerTok}[1]{{#1}}
    \newcommand{\ErrorTok}[1]{\textcolor[rgb]{1.00,0.00,0.00}{\textbf{{#1}}}}
    \newcommand{\NormalTok}[1]{{#1}}
    
    % Additional commands for more recent versions of Pandoc
    \newcommand{\ConstantTok}[1]{\textcolor[rgb]{0.53,0.00,0.00}{{#1}}}
    \newcommand{\SpecialCharTok}[1]{\textcolor[rgb]{0.25,0.44,0.63}{{#1}}}
    \newcommand{\VerbatimStringTok}[1]{\textcolor[rgb]{0.25,0.44,0.63}{{#1}}}
    \newcommand{\SpecialStringTok}[1]{\textcolor[rgb]{0.73,0.40,0.53}{{#1}}}
    \newcommand{\ImportTok}[1]{{#1}}
    \newcommand{\DocumentationTok}[1]{\textcolor[rgb]{0.73,0.13,0.13}{\textit{{#1}}}}
    \newcommand{\AnnotationTok}[1]{\textcolor[rgb]{0.38,0.63,0.69}{\textbf{\textit{{#1}}}}}
    \newcommand{\CommentVarTok}[1]{\textcolor[rgb]{0.38,0.63,0.69}{\textbf{\textit{{#1}}}}}
    \newcommand{\VariableTok}[1]{\textcolor[rgb]{0.10,0.09,0.49}{{#1}}}
    \newcommand{\ControlFlowTok}[1]{\textcolor[rgb]{0.00,0.44,0.13}{\textbf{{#1}}}}
    \newcommand{\OperatorTok}[1]{\textcolor[rgb]{0.40,0.40,0.40}{{#1}}}
    \newcommand{\BuiltInTok}[1]{{#1}}
    \newcommand{\ExtensionTok}[1]{{#1}}
    \newcommand{\PreprocessorTok}[1]{\textcolor[rgb]{0.74,0.48,0.00}{{#1}}}
    \newcommand{\AttributeTok}[1]{\textcolor[rgb]{0.49,0.56,0.16}{{#1}}}
    \newcommand{\InformationTok}[1]{\textcolor[rgb]{0.38,0.63,0.69}{\textbf{\textit{{#1}}}}}
    \newcommand{\WarningTok}[1]{\textcolor[rgb]{0.38,0.63,0.69}{\textbf{\textit{{#1}}}}}
    
    
    % Define a nice break command that doesn't care if a line doesn't already
    % exist.
    \def\br{\hspace*{\fill} \\* }
    % Math Jax compatability definitions
    \def\gt{>}
    \def\lt{<}
    % Document parameters
    \title{CSTLEngin\_HW3}
    
    
    

    % Pygments definitions
    
\makeatletter
\def\PY@reset{\let\PY@it=\relax \let\PY@bf=\relax%
    \let\PY@ul=\relax \let\PY@tc=\relax%
    \let\PY@bc=\relax \let\PY@ff=\relax}
\def\PY@tok#1{\csname PY@tok@#1\endcsname}
\def\PY@toks#1+{\ifx\relax#1\empty\else%
    \PY@tok{#1}\expandafter\PY@toks\fi}
\def\PY@do#1{\PY@bc{\PY@tc{\PY@ul{%
    \PY@it{\PY@bf{\PY@ff{#1}}}}}}}
\def\PY#1#2{\PY@reset\PY@toks#1+\relax+\PY@do{#2}}

\expandafter\def\csname PY@tok@w\endcsname{\def\PY@tc##1{\textcolor[rgb]{0.73,0.73,0.73}{##1}}}
\expandafter\def\csname PY@tok@c\endcsname{\let\PY@it=\textit\def\PY@tc##1{\textcolor[rgb]{0.25,0.50,0.50}{##1}}}
\expandafter\def\csname PY@tok@cp\endcsname{\def\PY@tc##1{\textcolor[rgb]{0.74,0.48,0.00}{##1}}}
\expandafter\def\csname PY@tok@k\endcsname{\let\PY@bf=\textbf\def\PY@tc##1{\textcolor[rgb]{0.00,0.50,0.00}{##1}}}
\expandafter\def\csname PY@tok@kp\endcsname{\def\PY@tc##1{\textcolor[rgb]{0.00,0.50,0.00}{##1}}}
\expandafter\def\csname PY@tok@kt\endcsname{\def\PY@tc##1{\textcolor[rgb]{0.69,0.00,0.25}{##1}}}
\expandafter\def\csname PY@tok@o\endcsname{\def\PY@tc##1{\textcolor[rgb]{0.40,0.40,0.40}{##1}}}
\expandafter\def\csname PY@tok@ow\endcsname{\let\PY@bf=\textbf\def\PY@tc##1{\textcolor[rgb]{0.67,0.13,1.00}{##1}}}
\expandafter\def\csname PY@tok@nb\endcsname{\def\PY@tc##1{\textcolor[rgb]{0.00,0.50,0.00}{##1}}}
\expandafter\def\csname PY@tok@nf\endcsname{\def\PY@tc##1{\textcolor[rgb]{0.00,0.00,1.00}{##1}}}
\expandafter\def\csname PY@tok@nc\endcsname{\let\PY@bf=\textbf\def\PY@tc##1{\textcolor[rgb]{0.00,0.00,1.00}{##1}}}
\expandafter\def\csname PY@tok@nn\endcsname{\let\PY@bf=\textbf\def\PY@tc##1{\textcolor[rgb]{0.00,0.00,1.00}{##1}}}
\expandafter\def\csname PY@tok@ne\endcsname{\let\PY@bf=\textbf\def\PY@tc##1{\textcolor[rgb]{0.82,0.25,0.23}{##1}}}
\expandafter\def\csname PY@tok@nv\endcsname{\def\PY@tc##1{\textcolor[rgb]{0.10,0.09,0.49}{##1}}}
\expandafter\def\csname PY@tok@no\endcsname{\def\PY@tc##1{\textcolor[rgb]{0.53,0.00,0.00}{##1}}}
\expandafter\def\csname PY@tok@nl\endcsname{\def\PY@tc##1{\textcolor[rgb]{0.63,0.63,0.00}{##1}}}
\expandafter\def\csname PY@tok@ni\endcsname{\let\PY@bf=\textbf\def\PY@tc##1{\textcolor[rgb]{0.60,0.60,0.60}{##1}}}
\expandafter\def\csname PY@tok@na\endcsname{\def\PY@tc##1{\textcolor[rgb]{0.49,0.56,0.16}{##1}}}
\expandafter\def\csname PY@tok@nt\endcsname{\let\PY@bf=\textbf\def\PY@tc##1{\textcolor[rgb]{0.00,0.50,0.00}{##1}}}
\expandafter\def\csname PY@tok@nd\endcsname{\def\PY@tc##1{\textcolor[rgb]{0.67,0.13,1.00}{##1}}}
\expandafter\def\csname PY@tok@s\endcsname{\def\PY@tc##1{\textcolor[rgb]{0.73,0.13,0.13}{##1}}}
\expandafter\def\csname PY@tok@sd\endcsname{\let\PY@it=\textit\def\PY@tc##1{\textcolor[rgb]{0.73,0.13,0.13}{##1}}}
\expandafter\def\csname PY@tok@si\endcsname{\let\PY@bf=\textbf\def\PY@tc##1{\textcolor[rgb]{0.73,0.40,0.53}{##1}}}
\expandafter\def\csname PY@tok@se\endcsname{\let\PY@bf=\textbf\def\PY@tc##1{\textcolor[rgb]{0.73,0.40,0.13}{##1}}}
\expandafter\def\csname PY@tok@sr\endcsname{\def\PY@tc##1{\textcolor[rgb]{0.73,0.40,0.53}{##1}}}
\expandafter\def\csname PY@tok@ss\endcsname{\def\PY@tc##1{\textcolor[rgb]{0.10,0.09,0.49}{##1}}}
\expandafter\def\csname PY@tok@sx\endcsname{\def\PY@tc##1{\textcolor[rgb]{0.00,0.50,0.00}{##1}}}
\expandafter\def\csname PY@tok@m\endcsname{\def\PY@tc##1{\textcolor[rgb]{0.40,0.40,0.40}{##1}}}
\expandafter\def\csname PY@tok@gh\endcsname{\let\PY@bf=\textbf\def\PY@tc##1{\textcolor[rgb]{0.00,0.00,0.50}{##1}}}
\expandafter\def\csname PY@tok@gu\endcsname{\let\PY@bf=\textbf\def\PY@tc##1{\textcolor[rgb]{0.50,0.00,0.50}{##1}}}
\expandafter\def\csname PY@tok@gd\endcsname{\def\PY@tc##1{\textcolor[rgb]{0.63,0.00,0.00}{##1}}}
\expandafter\def\csname PY@tok@gi\endcsname{\def\PY@tc##1{\textcolor[rgb]{0.00,0.63,0.00}{##1}}}
\expandafter\def\csname PY@tok@gr\endcsname{\def\PY@tc##1{\textcolor[rgb]{1.00,0.00,0.00}{##1}}}
\expandafter\def\csname PY@tok@ge\endcsname{\let\PY@it=\textit}
\expandafter\def\csname PY@tok@gs\endcsname{\let\PY@bf=\textbf}
\expandafter\def\csname PY@tok@gp\endcsname{\let\PY@bf=\textbf\def\PY@tc##1{\textcolor[rgb]{0.00,0.00,0.50}{##1}}}
\expandafter\def\csname PY@tok@go\endcsname{\def\PY@tc##1{\textcolor[rgb]{0.53,0.53,0.53}{##1}}}
\expandafter\def\csname PY@tok@gt\endcsname{\def\PY@tc##1{\textcolor[rgb]{0.00,0.27,0.87}{##1}}}
\expandafter\def\csname PY@tok@err\endcsname{\def\PY@bc##1{\setlength{\fboxsep}{0pt}\fcolorbox[rgb]{1.00,0.00,0.00}{1,1,1}{\strut ##1}}}
\expandafter\def\csname PY@tok@kc\endcsname{\let\PY@bf=\textbf\def\PY@tc##1{\textcolor[rgb]{0.00,0.50,0.00}{##1}}}
\expandafter\def\csname PY@tok@kd\endcsname{\let\PY@bf=\textbf\def\PY@tc##1{\textcolor[rgb]{0.00,0.50,0.00}{##1}}}
\expandafter\def\csname PY@tok@kn\endcsname{\let\PY@bf=\textbf\def\PY@tc##1{\textcolor[rgb]{0.00,0.50,0.00}{##1}}}
\expandafter\def\csname PY@tok@kr\endcsname{\let\PY@bf=\textbf\def\PY@tc##1{\textcolor[rgb]{0.00,0.50,0.00}{##1}}}
\expandafter\def\csname PY@tok@bp\endcsname{\def\PY@tc##1{\textcolor[rgb]{0.00,0.50,0.00}{##1}}}
\expandafter\def\csname PY@tok@fm\endcsname{\def\PY@tc##1{\textcolor[rgb]{0.00,0.00,1.00}{##1}}}
\expandafter\def\csname PY@tok@vc\endcsname{\def\PY@tc##1{\textcolor[rgb]{0.10,0.09,0.49}{##1}}}
\expandafter\def\csname PY@tok@vg\endcsname{\def\PY@tc##1{\textcolor[rgb]{0.10,0.09,0.49}{##1}}}
\expandafter\def\csname PY@tok@vi\endcsname{\def\PY@tc##1{\textcolor[rgb]{0.10,0.09,0.49}{##1}}}
\expandafter\def\csname PY@tok@vm\endcsname{\def\PY@tc##1{\textcolor[rgb]{0.10,0.09,0.49}{##1}}}
\expandafter\def\csname PY@tok@sa\endcsname{\def\PY@tc##1{\textcolor[rgb]{0.73,0.13,0.13}{##1}}}
\expandafter\def\csname PY@tok@sb\endcsname{\def\PY@tc##1{\textcolor[rgb]{0.73,0.13,0.13}{##1}}}
\expandafter\def\csname PY@tok@sc\endcsname{\def\PY@tc##1{\textcolor[rgb]{0.73,0.13,0.13}{##1}}}
\expandafter\def\csname PY@tok@dl\endcsname{\def\PY@tc##1{\textcolor[rgb]{0.73,0.13,0.13}{##1}}}
\expandafter\def\csname PY@tok@s2\endcsname{\def\PY@tc##1{\textcolor[rgb]{0.73,0.13,0.13}{##1}}}
\expandafter\def\csname PY@tok@sh\endcsname{\def\PY@tc##1{\textcolor[rgb]{0.73,0.13,0.13}{##1}}}
\expandafter\def\csname PY@tok@s1\endcsname{\def\PY@tc##1{\textcolor[rgb]{0.73,0.13,0.13}{##1}}}
\expandafter\def\csname PY@tok@mb\endcsname{\def\PY@tc##1{\textcolor[rgb]{0.40,0.40,0.40}{##1}}}
\expandafter\def\csname PY@tok@mf\endcsname{\def\PY@tc##1{\textcolor[rgb]{0.40,0.40,0.40}{##1}}}
\expandafter\def\csname PY@tok@mh\endcsname{\def\PY@tc##1{\textcolor[rgb]{0.40,0.40,0.40}{##1}}}
\expandafter\def\csname PY@tok@mi\endcsname{\def\PY@tc##1{\textcolor[rgb]{0.40,0.40,0.40}{##1}}}
\expandafter\def\csname PY@tok@il\endcsname{\def\PY@tc##1{\textcolor[rgb]{0.40,0.40,0.40}{##1}}}
\expandafter\def\csname PY@tok@mo\endcsname{\def\PY@tc##1{\textcolor[rgb]{0.40,0.40,0.40}{##1}}}
\expandafter\def\csname PY@tok@ch\endcsname{\let\PY@it=\textit\def\PY@tc##1{\textcolor[rgb]{0.25,0.50,0.50}{##1}}}
\expandafter\def\csname PY@tok@cm\endcsname{\let\PY@it=\textit\def\PY@tc##1{\textcolor[rgb]{0.25,0.50,0.50}{##1}}}
\expandafter\def\csname PY@tok@cpf\endcsname{\let\PY@it=\textit\def\PY@tc##1{\textcolor[rgb]{0.25,0.50,0.50}{##1}}}
\expandafter\def\csname PY@tok@c1\endcsname{\let\PY@it=\textit\def\PY@tc##1{\textcolor[rgb]{0.25,0.50,0.50}{##1}}}
\expandafter\def\csname PY@tok@cs\endcsname{\let\PY@it=\textit\def\PY@tc##1{\textcolor[rgb]{0.25,0.50,0.50}{##1}}}

\def\PYZbs{\char`\\}
\def\PYZus{\char`\_}
\def\PYZob{\char`\{}
\def\PYZcb{\char`\}}
\def\PYZca{\char`\^}
\def\PYZam{\char`\&}
\def\PYZlt{\char`\<}
\def\PYZgt{\char`\>}
\def\PYZsh{\char`\#}
\def\PYZpc{\char`\%}
\def\PYZdl{\char`\$}
\def\PYZhy{\char`\-}
\def\PYZsq{\char`\'}
\def\PYZdq{\char`\"}
\def\PYZti{\char`\~}
% for compatibility with earlier versions
\def\PYZat{@}
\def\PYZlb{[}
\def\PYZrb{]}
\makeatother


    % Exact colors from NB
    \definecolor{incolor}{rgb}{0.0, 0.0, 0.5}
    \definecolor{outcolor}{rgb}{0.545, 0.0, 0.0}



    
    % Prevent overflowing lines due to hard-to-break entities
    \sloppy 
    % Setup hyperref package
    \hypersetup{
      breaklinks=true,  % so long urls are correctly broken across lines
      colorlinks=true,
      urlcolor=urlcolor,
      linkcolor=linkcolor,
      citecolor=citecolor,
      }
    % Slightly bigger margins than the latex defaults
    
    \geometry{verbose,tmargin=1in,bmargin=1in,lmargin=1in,rmargin=1in}
    
    

    \begin{document}
    
    
    \maketitle
    
    

    
    Coastal Engineering - Homework 1

    Honor Code Pledge: ``I have neither given nor received unauthorized
assistance on this assignment.''

Signature:

    \begin{Verbatim}[commandchars=\\\{\}]
{\color{incolor}In [{\color{incolor}94}]:} \PY{k+kn}{import} \PY{n+nn}{pandas} \PY{k}{as} \PY{n+nn}{pd}
         \PY{k+kn}{import} \PY{n+nn}{math}
         \PY{k+kn}{import} \PY{n+nn}{numpy} \PY{k}{as} \PY{n+nn}{np}
         \PY{k+kn}{import} \PY{n+nn}{seaborn} \PY{k}{as} \PY{n+nn}{sns}
         \PY{k+kn}{import} \PY{n+nn}{matplotlib}\PY{n+nn}{.}\PY{n+nn}{pyplot} \PY{k}{as} \PY{n+nn}{plt}
         
         \PY{n}{g} \PY{o}{=} \PY{l+m+mf}{9.81} \PY{c+c1}{\PYZsh{} m/s\PYZca{}2}
         \PY{n}{pi} \PY{o}{=} \PY{n}{math}\PY{o}{.}\PY{n}{pi}
\end{Verbatim}


    Problem 1

    Given: wave heights (m)

Find: Calculate Hmax, H1/10, H1/3, Hmean, and Hrms; estimate with the
Rayleigh probaiblity distribtion H1/10, Hmean, and Hrms. Find Error.

Principles:

    \begin{Verbatim}[commandchars=\\\{\}]
{\color{incolor}In [{\color{incolor}95}]:} \PY{n}{data} \PY{o}{=} \PY{n}{pd}\PY{o}{.}\PY{n}{read\PYZus{}table}\PY{p}{(}\PY{l+s+s2}{\PYZdq{}}\PY{l+s+s2}{HwavestatsF18.dat}\PY{l+s+s2}{\PYZdq{}}\PY{p}{,} \PY{n}{sep}\PY{o}{=}\PY{l+s+s2}{\PYZdq{}}\PY{l+s+s2}{\PYZbs{}}\PY{l+s+s2}{s+}\PY{l+s+s2}{\PYZdq{}}\PY{p}{,}  \PY{n}{names} \PY{o}{=} \PY{p}{[}\PY{l+s+s1}{\PYZsq{}}\PY{l+s+s1}{wv\PYZus{}ht\PYZus{}m}\PY{l+s+s1}{\PYZsq{}}\PY{p}{]}\PY{p}{)}
\end{Verbatim}


    \begin{Verbatim}[commandchars=\\\{\}]
{\color{incolor}In [{\color{incolor}96}]:} \PY{k}{def} \PY{n+nf}{statistics}\PY{p}{(}\PY{n}{data}\PY{p}{,} \PY{n}{column\PYZus{}name}\PY{p}{,} \PY{n}{rounding}\PY{p}{)}\PY{p}{:}
             \PY{n}{data}\PY{o}{.}\PY{n}{sort\PYZus{}values}\PY{p}{(}\PY{n}{column\PYZus{}name}\PY{p}{,} \PY{n}{ascending}\PY{o}{=}\PY{k+kc}{False}\PY{p}{,} \PY{n}{inplace}\PY{o}{=}\PY{k+kc}{True}\PY{p}{)} 
             \PY{c+c1}{\PYZsh{} sort values largest to smallest}
             
             \PY{n}{H\PYZus{}max} \PY{o}{=} \PY{n+nb}{round}\PY{p}{(}\PY{n}{data}\PY{p}{[}\PY{n}{column\PYZus{}name}\PY{p}{]}\PY{o}{.}\PY{n}{max}\PY{p}{(}\PY{p}{)}\PY{p}{,} \PY{n}{rounding}\PY{p}{)}
             \PY{n}{H\PYZus{}mean} \PY{o}{=} \PY{n+nb}{round}\PY{p}{(}\PY{n}{data}\PY{p}{[}\PY{n}{column\PYZus{}name}\PY{p}{]}\PY{o}{.}\PY{n}{mean}\PY{p}{(}\PY{p}{)}\PY{p}{,} \PY{n}{rounding}\PY{p}{)}
                        
             \PY{n}{count\PYZus{}row} \PY{o}{=} \PY{n}{data}\PY{o}{.}\PY{n}{shape}\PY{p}{[}\PY{l+m+mi}{0}\PY{p}{]}
             \PY{n}{count\PYZus{}row}
             \PY{n}{count\PYZus{}third} \PY{o}{=} \PY{n+nb}{int}\PY{p}{(}\PY{n+nb}{round}\PY{p}{(}\PY{p}{(}\PY{n}{count\PYZus{}row} \PY{o}{*} \PY{p}{(}\PY{l+m+mi}{1}\PY{o}{/}\PY{l+m+mi}{3}\PY{p}{)}\PY{p}{)}\PY{p}{,} \PY{l+m+mi}{0}\PY{p}{)} \PY{o}{\PYZhy{}} \PY{l+m+mi}{1}\PY{p}{)} 
             \PY{c+c1}{\PYZsh{}Python indexing starts at 0, so must substract 1 from the count}
             \PY{n}{count\PYZus{}tenth} \PY{o}{=} \PY{n+nb}{int}\PY{p}{(}\PY{n+nb}{round}\PY{p}{(}\PY{p}{(}\PY{n}{count\PYZus{}row} \PY{o}{*} \PY{p}{(}\PY{l+m+mi}{1}\PY{o}{/}\PY{l+m+mi}{10}\PY{p}{)}\PY{p}{)}\PY{p}{,} \PY{l+m+mi}{0}\PY{p}{)} \PY{o}{\PYZhy{}} \PY{l+m+mi}{1}\PY{p}{)} 
             \PY{c+c1}{\PYZsh{}Python indexing starts at 0, so must substract 1 from the count}
                        
             \PY{n}{H\PYZus{}third} \PY{o}{=} \PY{n}{np}\PY{o}{.}\PY{n}{float}\PY{p}{(}\PY{n+nb}{round}\PY{p}{(}\PY{n}{data}\PY{o}{.}\PY{n}{iloc}\PY{p}{[}\PY{l+m+mi}{0}\PY{p}{:}\PY{n}{count\PYZus{}third}\PY{p}{]}\PY{o}{.}\PY{n}{mean}\PY{p}{(}\PY{p}{)}\PY{p}{,} \PY{n}{rounding}\PY{p}{)}\PY{p}{)}
             \PY{n}{H\PYZus{}tenth} \PY{o}{=} \PY{n}{np}\PY{o}{.}\PY{n}{float}\PY{p}{(}\PY{n+nb}{round}\PY{p}{(}\PY{n}{data}\PY{o}{.}\PY{n}{iloc}\PY{p}{[}\PY{l+m+mi}{0}\PY{p}{:}\PY{n}{count\PYZus{}tenth}\PY{p}{]}\PY{o}{.}\PY{n}{mean}\PY{p}{(}\PY{p}{)}\PY{p}{,} \PY{n}{rounding}\PY{p}{)}\PY{p}{)}
             
             \PY{n}{H\PYZus{}rms} \PY{o}{=} \PY{n+nb}{round}\PY{p}{(}\PY{n}{np}\PY{o}{.}\PY{n}{sqrt}\PY{p}{(}\PY{n}{np}\PY{o}{.}\PY{n}{mean}\PY{p}{(}\PY{n}{data}\PY{p}{[}\PY{l+s+s1}{\PYZsq{}}\PY{l+s+s1}{wv\PYZus{}ht\PYZus{}m}\PY{l+s+s1}{\PYZsq{}}\PY{p}{]}\PY{o}{*}\PY{o}{*}\PY{l+m+mi}{2}\PY{p}{)}\PY{p}{)}\PY{p}{,} \PY{n}{rounding}\PY{p}{)}
             
             \PY{k}{return} \PY{n}{H\PYZus{}max}\PY{p}{,} \PY{n}{H\PYZus{}mean}\PY{p}{,} \PY{n}{H\PYZus{}third}\PY{p}{,} \PY{n}{H\PYZus{}tenth}\PY{p}{,} \PY{n}{H\PYZus{}rms}
\end{Verbatim}


    \begin{Verbatim}[commandchars=\\\{\}]
{\color{incolor}In [{\color{incolor}97}]:} \PY{n}{H\PYZus{}max}\PY{p}{,} \PY{n}{H\PYZus{}mean}\PY{p}{,} \PY{n}{H\PYZus{}third}\PY{p}{,} \PY{n}{H\PYZus{}tenth}\PY{p}{,} \PY{n}{H\PYZus{}rms} \PY{o}{=} \PY{n}{statistics}\PY{p}{(}\PY{n}{data}\PY{p}{,} \PY{l+s+s1}{\PYZsq{}}\PY{l+s+s1}{wv\PYZus{}ht\PYZus{}m}\PY{l+s+s1}{\PYZsq{}}\PY{p}{,} \PY{l+m+mi}{2}\PY{p}{)}
         \PY{n+nb}{print}\PY{p}{(}\PY{l+s+s1}{\PYZsq{}}\PY{l+s+s1}{H\PYZus{}max is }\PY{l+s+si}{\PYZob{}\PYZcb{}}\PY{l+s+s1}{ m}\PY{l+s+s1}{\PYZsq{}}\PY{o}{.}\PY{n}{format}\PY{p}{(}\PY{n}{H\PYZus{}max}\PY{p}{)}\PY{p}{)}
         \PY{n+nb}{print}\PY{p}{(}\PY{l+s+s1}{\PYZsq{}}\PY{l+s+s1}{H\PYZus{}1/10 is }\PY{l+s+si}{\PYZob{}\PYZcb{}}\PY{l+s+s1}{ m}\PY{l+s+s1}{\PYZsq{}}\PY{o}{.}\PY{n}{format}\PY{p}{(}\PY{n}{H\PYZus{}tenth}\PY{p}{)}\PY{p}{)}
         \PY{n+nb}{print}\PY{p}{(}\PY{l+s+s1}{\PYZsq{}}\PY{l+s+s1}{H\PYZus{}1/3 is }\PY{l+s+si}{\PYZob{}\PYZcb{}}\PY{l+s+s1}{ m}\PY{l+s+s1}{\PYZsq{}}\PY{o}{.}\PY{n}{format}\PY{p}{(}\PY{n}{H\PYZus{}third}\PY{p}{)}\PY{p}{)}
         \PY{n+nb}{print}\PY{p}{(}\PY{l+s+s1}{\PYZsq{}}\PY{l+s+s1}{H\PYZus{}mean is }\PY{l+s+si}{\PYZob{}\PYZcb{}}\PY{l+s+s1}{ m}\PY{l+s+s1}{\PYZsq{}}\PY{o}{.}\PY{n}{format}\PY{p}{(}\PY{n}{H\PYZus{}mean}\PY{p}{)}\PY{p}{)}
         \PY{n+nb}{print}\PY{p}{(}\PY{l+s+s1}{\PYZsq{}}\PY{l+s+s1}{H\PYZus{}rms is }\PY{l+s+si}{\PYZob{}\PYZcb{}}\PY{l+s+s1}{ m}\PY{l+s+s1}{\PYZsq{}}\PY{o}{.}\PY{n}{format}\PY{p}{(}\PY{n}{H\PYZus{}rms}\PY{p}{)}\PY{p}{)}
\end{Verbatim}


    \begin{Verbatim}[commandchars=\\\{\}]
H\_max is 12.42 m
H\_1/10 is 4.54 m
H\_1/3 is 3.47 m
H\_mean is 1.97 m
H\_rms is 2.37 m

    \end{Verbatim}

    \begin{Verbatim}[commandchars=\\\{\}]
{\color{incolor}In [{\color{incolor}98}]:} \PY{k}{def} \PY{n+nf}{Rayleigh\PYZus{}estimate}\PY{p}{(}\PY{n}{H\PYZus{}third}\PY{p}{,} \PY{n}{rounding}\PY{p}{)}\PY{p}{:}
             \PY{n}{est\PYZus{}H\PYZus{}tenth} \PY{o}{=} \PY{n}{np}\PY{o}{.}\PY{n}{round}\PY{p}{(}\PY{p}{(}\PY{l+m+mf}{1.27} \PY{o}{*} \PY{p}{(}\PY{n}{H\PYZus{}third}\PY{p}{)}\PY{p}{)}\PY{p}{,} \PY{n}{rounding}\PY{p}{)}
             \PY{n}{est\PYZus{}H\PYZus{}mean} \PY{o}{=} \PY{n}{np}\PY{o}{.}\PY{n}{round}\PY{p}{(}\PY{p}{(}\PY{l+m+mf}{1.27} \PY{o}{*} \PY{p}{(}\PY{n}{H\PYZus{}third}\PY{p}{)} \PY{o}{/} \PY{l+m+mf}{2.03}\PY{p}{)}\PY{p}{,} \PY{n}{rounding}\PY{p}{)}
             \PY{n}{est\PYZus{}H\PYZus{}rms} \PY{o}{=} \PY{n}{np}\PY{o}{.}\PY{n}{round}\PY{p}{(}\PY{p}{(}\PY{l+m+mf}{1.27} \PY{o}{*} \PY{p}{(}\PY{n}{H\PYZus{}third}\PY{p}{)} \PY{o}{/} \PY{l+m+mf}{1.80}\PY{p}{)}\PY{p}{,} \PY{n}{rounding}\PY{p}{)}
             
             \PY{k}{return} \PY{n}{est\PYZus{}H\PYZus{}tenth}\PY{p}{,} \PY{n}{est\PYZus{}H\PYZus{}mean}\PY{p}{,} \PY{n}{est\PYZus{}H\PYZus{}rms}  
\end{Verbatim}


    \begin{Verbatim}[commandchars=\\\{\}]
{\color{incolor}In [{\color{incolor}99}]:} \PY{n}{est\PYZus{}H\PYZus{}tenth}\PY{p}{,} \PY{n}{est\PYZus{}H\PYZus{}mean}\PY{p}{,} \PY{n}{est\PYZus{}H\PYZus{}rms} \PY{o}{=} \PY{n}{Rayleigh\PYZus{}estimate}\PY{p}{(}\PY{n}{H\PYZus{}third}\PY{p}{,} \PY{l+m+mi}{2}\PY{p}{)}
\end{Verbatim}


    \begin{Verbatim}[commandchars=\\\{\}]
{\color{incolor}In [{\color{incolor}100}]:} \PY{k}{def} \PY{n+nf}{Errors}\PY{p}{(}\PY{n}{true}\PY{p}{,} \PY{n}{estimate}\PY{p}{,} \PY{n}{rounding}\PY{p}{)}\PY{p}{:}
              \PY{n}{error} \PY{o}{=} \PY{n}{np}\PY{o}{.}\PY{n}{round}\PY{p}{(}\PY{p}{(}\PY{n+nb}{abs}\PY{p}{(}\PY{n}{true} \PY{o}{\PYZhy{}} \PY{n}{estimate}\PY{p}{)}\PY{o}{/}\PY{n}{true} \PY{o}{*} \PY{l+m+mi}{100}\PY{p}{)}\PY{p}{,} \PY{n}{rounding}\PY{p}{)}
              
              \PY{k}{return} \PY{n}{error}
\end{Verbatim}


    \begin{Verbatim}[commandchars=\\\{\}]
{\color{incolor}In [{\color{incolor}101}]:} \PY{n}{error\PYZus{}H\PYZus{}tenth} \PY{o}{=} \PY{n}{Errors}\PY{p}{(}\PY{n}{H\PYZus{}tenth}\PY{p}{,} \PY{n}{est\PYZus{}H\PYZus{}tenth}\PY{p}{,} \PY{l+m+mi}{2}\PY{p}{)}
          \PY{n}{error\PYZus{}H\PYZus{}mean} \PY{o}{=} \PY{n}{Errors}\PY{p}{(}\PY{n}{H\PYZus{}mean}\PY{p}{,} \PY{n}{est\PYZus{}H\PYZus{}mean}\PY{p}{,} \PY{l+m+mi}{2}\PY{p}{)}
          \PY{n}{error\PYZus{}H\PYZus{}rms} \PY{o}{=} \PY{n}{Errors}\PY{p}{(}\PY{n}{H\PYZus{}rms}\PY{p}{,} \PY{n}{est\PYZus{}H\PYZus{}rms}\PY{p}{,} \PY{l+m+mi}{2}\PY{p}{)}
\end{Verbatim}


    \begin{Verbatim}[commandchars=\\\{\}]
{\color{incolor}In [{\color{incolor}102}]:} \PY{n+nb}{print}\PY{p}{(}\PY{l+s+s1}{\PYZsq{}}\PY{l+s+s1}{H\PYZus{}1/10 estimated from Rayleigh probability distribution is }\PY{l+s+si}{\PYZob{}\PYZcb{}}\PY{l+s+s1}{ m }\PY{l+s+se}{\PYZbs{}n}\PY{l+s+s1}{ with an error of }\PY{l+s+si}{\PYZob{}\PYZcb{}}\PY{l+s+s1}{\PYZpc{}}\PY{l+s+s1}{\PYZsq{}}\PY{o}{.}\PY{n}{format}\PY{p}{(}\PY{n}{est\PYZus{}H\PYZus{}tenth}\PY{p}{,} \PY{n}{error\PYZus{}H\PYZus{}tenth}\PY{p}{)}\PY{p}{)}
          \PY{n+nb}{print}\PY{p}{(}\PY{l+s+s1}{\PYZsq{}}\PY{l+s+s1}{H\PYZus{}mean estimated from Rayleigh probability distribution is }\PY{l+s+si}{\PYZob{}\PYZcb{}}\PY{l+s+s1}{ m }\PY{l+s+se}{\PYZbs{}n}\PY{l+s+s1}{ with an error of }\PY{l+s+si}{\PYZob{}\PYZcb{}}\PY{l+s+s1}{\PYZpc{}}\PY{l+s+s1}{\PYZsq{}}\PY{o}{.}\PY{n}{format}\PY{p}{(}\PY{n}{est\PYZus{}H\PYZus{}mean}\PY{p}{,} \PY{n}{error\PYZus{}H\PYZus{}mean}\PY{p}{)}\PY{p}{)}
          \PY{n+nb}{print}\PY{p}{(}\PY{l+s+s1}{\PYZsq{}}\PY{l+s+s1}{H\PYZus{}rms estimated from Rayleigh probability distribution is }\PY{l+s+si}{\PYZob{}\PYZcb{}}\PY{l+s+s1}{ m }\PY{l+s+se}{\PYZbs{}n}\PY{l+s+s1}{ with an error of }\PY{l+s+si}{\PYZob{}\PYZcb{}}\PY{l+s+s1}{\PYZpc{}}\PY{l+s+s1}{\PYZsq{}}\PY{o}{.}\PY{n}{format}\PY{p}{(}\PY{n}{est\PYZus{}H\PYZus{}rms}\PY{p}{,} \PY{n}{error\PYZus{}H\PYZus{}rms}\PY{p}{)}\PY{p}{)}
\end{Verbatim}


    \begin{Verbatim}[commandchars=\\\{\}]
H\_1/10 estimated from Rayleigh probability distribution is 4.41 m 
 with an error of 2.86\%
H\_mean estimated from Rayleigh probability distribution is 2.17 m 
 with an error of 10.15\%
H\_rms estimated from Rayleigh probability distribution is 2.45 m 
 with an error of 3.38\%

    \end{Verbatim}

    Discussion:\\
The Rayleigh prbability distribution underpredicted the H\_1/10 values
by 2.86\% but overpredicted the H\_mean and H\_rms values by 10.25\% and
3.38\% (respectively). This may mean that a more normal curve would work
better to estimate these values from the significant wave height. There
also may be an anomoly in wave height that is not accounted for in the
Rayleigh probability distribution, the H\_max is significantly higher
than the significant wave height (more than double) and, therefore,
calculations based on the significant wave height alone would
underpredicted the top 10\% of data.

    

    Problem 2

    Given: T = 16.0 s wave in a depth of h = 6.50 m with a bottom slope
\textrm{tan}\betatan β = 0.040; Ks = 1.25

Find: (s) breaking wave height, Hb with Weggel 1972, compare with k
assumed to be 0.78; (b) Type of breaker; (c) volume flux at breaking

Principles:

    \begin{enumerate}
\def\labelenumi{(\alph{enumi})}
\item
\end{enumerate}

    \begin{Verbatim}[commandchars=\\\{\}]
{\color{incolor}In [{\color{incolor}103}]:} \PY{n}{T} \PY{o}{=} \PY{l+m+mf}{16.0} \PY{c+c1}{\PYZsh{}1/s}
          \PY{n}{h} \PY{o}{=} \PY{l+m+mf}{6.50} \PY{c+c1}{\PYZsh{}m}
          \PY{n}{tan\PYZus{}beta} \PY{o}{=} \PY{l+m+mf}{0.040}
          
          \PY{n}{a} \PY{o}{=} \PY{l+m+mf}{43.8} \PY{o}{*} \PY{p}{(}\PY{l+m+mi}{1} \PY{o}{\PYZhy{}} \PY{n}{np}\PY{o}{.}\PY{n}{exp}\PY{p}{(}\PY{o}{\PYZhy{}}\PY{l+m+mi}{19} \PY{o}{*} \PY{n}{tan\PYZus{}beta}\PY{p}{)}\PY{p}{)}
          \PY{n}{b} \PY{o}{=} \PY{p}{(}\PY{l+m+mf}{1.56}\PY{p}{)} \PY{o}{/} \PY{p}{(}\PY{l+m+mi}{1} \PY{o}{+} \PY{n}{np}\PY{o}{.}\PY{n}{exp}\PY{p}{(}\PY{o}{\PYZhy{}}\PY{l+m+mf}{19.5} \PY{o}{*} \PY{n}{tan\PYZus{}beta}\PY{p}{)}\PY{p}{)}
\end{Verbatim}


    \begin{Verbatim}[commandchars=\\\{\}]
{\color{incolor}In [{\color{incolor}104}]:} \PY{n}{sol} \PY{o}{=} \PY{p}{[}\PY{l+m+mi}{0}\PY{p}{]}
          \PY{n}{solindex} \PY{o}{=} \PY{l+m+mi}{0}
          
          \PY{n}{iteration} \PY{o}{=} \PY{l+m+mi}{0}
          \PY{n}{initial\PYZus{}guess} \PY{o}{=} \PY{l+m+mf}{0.78}
          \PY{c+c1}{\PYZsh{} input breaker index guess}
          \PY{n}{Kb} \PY{o}{=} \PY{n}{initial\PYZus{}guess}
          
          \PY{n}{\PYZus{}break} \PY{o}{=} \PY{k+kc}{True}
          
          \PY{k}{while} \PY{n}{\PYZus{}break} \PY{o}{==} \PY{k+kc}{True}\PY{p}{:} 
              \PY{n}{previous\PYZus{}guess} \PY{o}{=} \PY{n}{Kb}
              
              \PY{n}{Hb} \PY{o}{=} \PY{p}{(}\PY{n}{Kb} \PY{o}{*} \PY{l+m+mf}{6.50}\PY{p}{)}
              \PY{n}{Kb} \PY{o}{=} \PY{n}{b} \PY{o}{\PYZhy{}} \PY{n}{a}\PY{o}{*}\PY{p}{(}\PY{n}{Hb}\PY{o}{/}\PY{p}{(}\PY{n}{g}\PY{o}{*}\PY{n}{T}\PY{o}{*}\PY{o}{*}\PY{l+m+mi}{2}\PY{p}{)}\PY{p}{)}
          
              \PY{n}{error} \PY{o}{=} \PY{n+nb}{abs}\PY{p}{(}\PY{n}{Kb}\PY{o}{\PYZhy{}}\PY{n}{previous\PYZus{}guess}\PY{p}{)} 
              \PY{c+c1}{\PYZsh{} calculate error}
              \PY{k}{if} \PY{p}{(}\PY{n}{error} \PY{o}{\PYZlt{}}\PY{o}{=} \PY{l+m+mf}{0.000000001}\PY{p}{)} \PY{o+ow}{or} \PY{p}{(}\PY{n}{error} \PY{o}{==} \PY{l+m+mi}{0}\PY{p}{)}\PY{p}{:} 
                  \PY{c+c1}{\PYZsh{} break when the error threshold is met}
                  \PY{n}{\PYZus{}break} \PY{o}{=} \PY{k+kc}{False}
              \PY{n}{iteration} \PY{o}{+}\PY{o}{=} \PY{l+m+mi}{1}
          
          \PY{n}{sol}\PY{p}{[}\PY{n}{solindex}\PY{p}{]} \PY{o}{=} \PY{n}{Kb} 
          \PY{c+c1}{\PYZsh{}converged solution added}
          \PY{n}{solindex} \PY{o}{+}\PY{o}{=} \PY{l+m+mi}{1}
          
          \PY{n}{Kb\PYZus{}iter} \PY{o}{=} \PY{n+nb}{round}\PY{p}{(}\PY{n}{sol}\PY{p}{[}\PY{l+m+mi}{0}\PY{p}{]}\PY{p}{,} \PY{l+m+mi}{2}\PY{p}{)}
          \PY{n}{Hb\PYZus{}iter} \PY{o}{=} \PY{n+nb}{round}\PY{p}{(}\PY{p}{(}\PY{n}{Kb\PYZus{}iter} \PY{o}{*} \PY{l+m+mf}{6.50}\PY{p}{)}\PY{p}{,} \PY{l+m+mi}{2}\PY{p}{)}
\end{Verbatim}


    \begin{Verbatim}[commandchars=\\\{\}]
{\color{incolor}In [{\color{incolor}105}]:} \PY{n+nb}{print}\PY{p}{(}\PY{l+s+s1}{\PYZsq{}}\PY{l+s+s1}{Calculated with Weggel(1972), the breaking wave height (Hb) is }\PY{l+s+si}{\PYZob{}\PYZcb{}}\PY{l+s+s1}{ m }\PY{l+s+se}{\PYZbs{}n}\PY{l+s+s1}{ with a break point (Kb) value of }\PY{l+s+si}{\PYZob{}\PYZcb{}}\PY{l+s+s1}{\PYZsq{}}\PY{o}{.}\PY{n}{format}\PY{p}{(}\PY{n}{Hb\PYZus{}iter}\PY{p}{,} \PY{n}{Kb\PYZus{}iter}\PY{p}{)}\PY{p}{)}
\end{Verbatim}


    \begin{Verbatim}[commandchars=\\\{\}]
Calculated with Weggel(1972), the breaking wave height (Hb) is 6.56 m 
 with a break point (Kb) value of 1.01

    \end{Verbatim}

    \begin{Verbatim}[commandchars=\\\{\}]
{\color{incolor}In [{\color{incolor}106}]:} \PY{c+c1}{\PYZsh{}k = H/h}
          \PY{n}{H\PYZus{}assumed} \PY{o}{=} \PY{p}{(}\PY{l+m+mf}{0.78}\PY{p}{)} \PY{o}{*} \PY{n}{h}
          
          \PY{n+nb}{print}\PY{p}{(}\PY{l+s+s1}{\PYZsq{}}\PY{l+s+s1}{With an assumed break point (Kb) value of 0.78, }\PY{l+s+se}{\PYZbs{}n}\PY{l+s+s1}{ the calculated breaking wave height (Hb) is }\PY{l+s+si}{\PYZob{}\PYZcb{}}\PY{l+s+s1}{ m}\PY{l+s+s1}{\PYZsq{}}\PY{o}{.}\PY{n}{format}\PY{p}{(}\PY{n}{H\PYZus{}assumed}\PY{p}{)}\PY{p}{)}
\end{Verbatim}


    \begin{Verbatim}[commandchars=\\\{\}]
With an assumed break point (Kb) value of 0.78, 
 the calculated breaking wave height (Hb) is 5.07 m

    \end{Verbatim}

    \begin{Verbatim}[commandchars=\\\{\}]
{\color{incolor}In [{\color{incolor}107}]:} \PY{n}{Errors}\PY{p}{(}\PY{n}{Hb\PYZus{}iter}\PY{p}{,} \PY{n}{H\PYZus{}assumed}\PY{p}{,} \PY{l+m+mi}{2}\PY{p}{)}
\end{Verbatim}


\begin{Verbatim}[commandchars=\\\{\}]
{\color{outcolor}Out[{\color{outcolor}107}]:} 22.710000000000001
\end{Verbatim}
            
    Discussion:\\
The assumed breakpoint value that is often cited (0.78) produces an
underestimate of the breaking wave height by 22.7\%. This must be
because the 0.78 value is derived from a solitary wave with simple
conditions (no wave steepness and no slope incorporated), therefore
assuming a linear relationship between breaking wave height and local
water depth, indicating that the wave will break when its height is
approximately 0.8 of the depth. However, with a tan\_beta of 0.040
(slope of 2.3) the wave will steepen due to shoaling because of
shallower depth with the bottom slope. The Weggel empirical model
incorporates more nonlinearities, including wave steepness, which
increases as it enters shallower water with the sloping bathymetry.

    

    \begin{enumerate}
\def\labelenumi{(\alph{enumi})}
\setcounter{enumi}{1}
\item
\end{enumerate}

    \begin{Verbatim}[commandchars=\\\{\}]
{\color{incolor}In [{\color{incolor}108}]:} \PY{n}{Ks} \PY{o}{=} \PY{l+m+mf}{1.25}
          
          \PY{n}{L\PYZus{}0} \PY{o}{=} \PY{p}{(}\PY{p}{(}\PY{n}{g}\PY{o}{*}\PY{n}{T}\PY{o}{*}\PY{o}{*}\PY{l+m+mi}{2}\PY{p}{)}\PY{o}{/}\PY{p}{(}\PY{l+m+mi}{2}\PY{o}{*}\PY{n}{pi}\PY{p}{)}\PY{p}{)}
          \PY{n}{H\PYZus{}0} \PY{o}{=} \PY{n}{Hb\PYZus{}iter}\PY{o}{/}\PY{n}{Ks}
\end{Verbatim}


    \begin{Verbatim}[commandchars=\\\{\}]
{\color{incolor}In [{\color{incolor}109}]:} \PY{n}{surf\PYZus{}param} \PY{o}{=} \PY{n}{np}\PY{o}{.}\PY{n}{round}\PY{p}{(}\PY{n}{tan\PYZus{}beta}\PY{o}{/}\PY{p}{(}\PY{n}{np}\PY{o}{.}\PY{n}{sqrt}\PY{p}{(}\PY{n}{H\PYZus{}0}\PY{o}{/}\PY{n}{L\PYZus{}0}\PY{p}{)}\PY{p}{)}\PY{p}{,}\PY{l+m+mi}{3}\PY{p}{)}
          \PY{n+nb}{print}\PY{p}{(}\PY{l+s+s1}{\PYZsq{}}\PY{l+s+s1}{The surf similarity parameter is }\PY{l+s+si}{\PYZob{}\PYZcb{}}\PY{l+s+s1}{\PYZsq{}}\PY{o}{.}\PY{n}{format}\PY{p}{(}\PY{n}{surf\PYZus{}param}\PY{p}{)}\PY{p}{)}
\end{Verbatim}


    \begin{Verbatim}[commandchars=\\\{\}]
The surf similarity parameter is 0.349

    \end{Verbatim}

    With a surf similarity parameter of 0.35, this wave is a spilling
breaker. This type of breaking wave is characterized by a relative low
wave stpeeness and mild nearshore slope.

    

    \begin{enumerate}
\def\labelenumi{(\alph{enumi})}
\setcounter{enumi}{2}
\item
\end{enumerate}

    \begin{Verbatim}[commandchars=\\\{\}]
{\color{incolor}In [{\color{incolor}110}]:} \PY{n}{c} \PY{o}{=} \PY{n}{np}\PY{o}{.}\PY{n}{sqrt}\PY{p}{(}\PY{n}{g} \PY{o}{*} \PY{n}{h}\PY{p}{)} 
          \PY{c+c1}{\PYZsh{} m/s}
          \PY{n}{Q} \PY{o}{=} \PY{n}{np}\PY{o}{.}\PY{n}{round}\PY{p}{(}\PY{p}{(}\PY{n}{g} \PY{o}{*} \PY{p}{(}\PY{n}{Hb\PYZus{}iter}\PY{o}{*}\PY{o}{*}\PY{l+m+mi}{2}\PY{p}{)}\PY{p}{)}\PY{o}{/}\PY{p}{(}\PY{l+m+mi}{8}\PY{o}{*}\PY{n}{c}\PY{p}{)}\PY{p}{,}\PY{l+m+mi}{2}\PY{p}{)}
          \PY{n}{Q} \PY{c+c1}{\PYZsh{}m\PYZca{}2/s}
          \PY{n+nb}{print}\PY{p}{(}\PY{l+s+s1}{\PYZsq{}}\PY{l+s+s1}{The voume flux at the point of breaking is }\PY{l+s+si}{\PYZob{}\PYZcb{}}\PY{l+s+s1}{ m\PYZca{}2/s}\PY{l+s+s1}{\PYZsq{}}\PY{o}{.}\PY{n}{format}\PY{p}{(}\PY{n}{Q}\PY{p}{)}\PY{p}{)}
\end{Verbatim}


    \begin{Verbatim}[commandchars=\\\{\}]
The voume flux at the point of breaking is 6.61 m\^{}2/s

    \end{Verbatim}

    

    Problem 3

    Given: A normally incident (perpendicular) deep-water wave with
significant Hos = 2.90 m and T = 19.5 s propagates towards shore on a
bottom slope of \textrm{tan}\betatan β = 0.150. The wave breaks at a
depth of h = 5.40 m. Assuming the breaker index \kappa=\frac{H_b}{h_b}κ
= H b h b = 0.78 adequately estimates the breaking wave height

Find: (a) wave setup at the still-water shoreline; (b) Wave runup; (c)
Compare (a) and (b)

Principles:

    \begin{enumerate}
\def\labelenumi{(\alph{enumi})}
\item
\end{enumerate}

    \begin{Verbatim}[commandchars=\\\{\}]
{\color{incolor}In [{\color{incolor}111}]:} \PY{n}{H\PYZus{}sig} \PY{o}{=} \PY{l+m+mf}{2.90}
          \PY{n}{T} \PY{o}{=} \PY{l+m+mf}{19.5}
          \PY{n}{tan\PYZus{}beta} \PY{o}{=} \PY{l+m+mf}{0.150}
          \PY{n}{h\PYZus{}break} \PY{o}{=} \PY{l+m+mf}{5.40}
          \PY{n}{K\PYZus{}breaker} \PY{o}{=} \PY{l+m+mf}{0.78}
\end{Verbatim}


    \begin{Verbatim}[commandchars=\\\{\}]
{\color{incolor}In [{\color{incolor}112}]:} \PY{n}{x\PYZus{}shoreline} \PY{o}{=} \PY{n}{h\PYZus{}break} \PY{o}{/} \PY{n}{tan\PYZus{}beta}
          
          \PY{c+c1}{\PYZsh{} set up = \PYZhy{} (0.12 to 0.15)(\PYZhy{} tan\PYZus{}beta)(x\PYZus{}shoreline)}
          \PY{n}{s\PYZus{}1} \PY{o}{=} \PY{n+nb}{round}\PY{p}{(}\PY{p}{(}\PY{o}{\PYZhy{}} \PY{l+m+mf}{0.12}\PY{p}{)} \PY{o}{*} \PY{p}{(}\PY{o}{\PYZhy{}} \PY{n}{tan\PYZus{}beta}\PY{p}{)} \PY{o}{*} \PY{p}{(}\PY{n}{x\PYZus{}shoreline}\PY{p}{)}\PY{p}{,} \PY{l+m+mi}{3}\PY{p}{)}
          \PY{n}{s\PYZus{}2} \PY{o}{=} \PY{n+nb}{round}\PY{p}{(}\PY{p}{(}\PY{o}{\PYZhy{}} \PY{l+m+mf}{0.15}\PY{p}{)} \PY{o}{*} \PY{p}{(}\PY{o}{\PYZhy{}}\PY{n}{tan\PYZus{}beta}\PY{p}{)} \PY{o}{*} \PY{p}{(}\PY{n}{x\PYZus{}shoreline}\PY{p}{)}\PY{p}{,} \PY{l+m+mi}{3}\PY{p}{)}
          
          \PY{n+nb}{print}\PY{p}{(}\PY{l+s+s1}{\PYZsq{}}\PY{l+s+s1}{The wave setup at the still\PYZhy{}water shoreline }\PY{l+s+se}{\PYZbs{}n}\PY{l+s+s1}{ is between }\PY{l+s+si}{\PYZob{}\PYZcb{}}\PY{l+s+s1}{ to }\PY{l+s+si}{\PYZob{}\PYZcb{}}\PY{l+s+s1}{0 m}\PY{l+s+s1}{\PYZsq{}}\PY{o}{.}\PY{n}{format}\PY{p}{(}\PY{n}{s\PYZus{}1}\PY{p}{,} \PY{n}{s\PYZus{}2}\PY{p}{)}\PY{p}{)}
\end{Verbatim}


    \begin{Verbatim}[commandchars=\\\{\}]
The wave setup at the still-water shoreline 
 is between 0.648 to 0.810 m

    \end{Verbatim}

    \begin{enumerate}
\def\labelenumi{(\alph{enumi})}
\setcounter{enumi}{1}
\item
\end{enumerate}

    \begin{Verbatim}[commandchars=\\\{\}]
{\color{incolor}In [{\color{incolor}115}]:} \PY{n}{L\PYZus{}0} \PY{o}{=} \PY{p}{(}\PY{p}{(}\PY{n}{g}\PY{o}{*}\PY{n}{T}\PY{o}{*}\PY{o}{*}\PY{l+m+mi}{2}\PY{p}{)}\PY{o}{/}\PY{p}{(}\PY{l+m+mi}{2}\PY{o}{*}\PY{n}{pi}\PY{p}{)}\PY{p}{)}
          \PY{n}{L\PYZus{}1} \PY{o}{=} \PY{p}{(}\PY{p}{(}\PY{n}{g}\PY{o}{*}\PY{n}{T}\PY{o}{*}\PY{o}{*}\PY{l+m+mi}{2}\PY{p}{)}\PY{o}{/}\PY{p}{(}\PY{l+m+mi}{2}\PY{o}{*}\PY{n}{pi}\PY{p}{)}\PY{p}{)}\PY{o}{*} \PYZbs{}
          \PY{p}{(}\PY{n}{math}\PY{o}{.}\PY{n}{tanh}\PY{p}{(}\PY{p}{(}\PY{p}{(}\PY{p}{(}\PY{l+m+mi}{2}\PY{o}{*}\PY{n}{pi}\PY{p}{)}\PY{o}{/}\PY{n}{T}\PY{p}{)}\PY{o}{*}\PY{p}{(}\PY{p}{(}\PY{n}{h\PYZus{}break}\PY{o}{/}\PY{n}{g}\PY{p}{)}\PY{o}{*}\PY{o}{*}\PY{l+m+mf}{0.5}\PY{p}{)}\PY{p}{)}\PY{o}{*}\PY{o}{*}\PY{p}{(}\PY{l+m+mi}{3}\PY{o}{/}\PY{l+m+mi}{2}\PY{p}{)}\PY{p}{)}\PY{p}{)}\PY{o}{*}\PY{o}{*}\PY{p}{(}\PY{l+m+mi}{2}\PY{o}{/}\PY{l+m+mi}{3}\PY{p}{)} 
          \PY{c+c1}{\PYZsh{} L\PYZus{}1 Approximated with Fenton and McKee}
          
          \PY{n}{k\PYZus{}1} \PY{o}{=} \PY{p}{(}\PY{l+m+mi}{2}\PY{o}{*} \PY{n}{math}\PY{o}{.}\PY{n}{pi}\PY{p}{)} \PY{o}{/} \PY{n}{L\PYZus{}1}
          
          \PY{n}{N\PYZus{}0} \PY{o}{=} \PY{l+m+mf}{0.5}
          \PY{n}{N\PYZus{}1} \PY{o}{=} \PY{p}{(}\PY{l+m+mf}{0.5}\PY{p}{)}\PY{o}{*} \PY{p}{(}\PY{l+m+mi}{1} \PY{o}{+} \PY{p}{(}\PY{p}{(}\PY{l+m+mi}{2} \PY{o}{*} \PY{n}{k\PYZus{}1} \PY{o}{*} \PY{n}{h\PYZus{}break}\PY{p}{)} \PY{o}{/} \PYZbs{}
                             \PY{p}{(}\PY{n}{math}\PY{o}{.}\PY{n}{sinh}\PY{p}{(}\PY{l+m+mi}{2}\PY{o}{*}\PY{n}{k\PYZus{}1}\PY{o}{*}\PY{n}{h\PYZus{}break}\PY{p}{)}\PY{p}{)}\PY{p}{)}\PY{p}{)}
          
          \PY{n}{Ks} \PY{o}{=} \PY{n}{np}\PY{o}{.}\PY{n}{sqrt}\PY{p}{(}\PY{p}{(}\PY{n}{N\PYZus{}0} \PY{o}{*} \PY{n}{L\PYZus{}0}\PY{p}{)}\PY{o}{/}\PY{p}{(}\PY{n}{N\PYZus{}1} \PY{o}{*} \PY{n}{L\PYZus{}1}\PY{p}{)} \PY{p}{)}
          \PY{n}{H\PYZus{}1} \PY{o}{=} \PY{n}{K\PYZus{}breaker} \PY{o}{*} \PY{n}{h\PYZus{}break}
          \PY{n}{H\PYZus{}0} \PY{o}{=} \PY{n}{H\PYZus{}1} \PY{o}{/} \PY{n}{Ks}
          \PY{c+c1}{\PYZsh{} Shoaling to find H\PYZus{}0}
          
          
          \PY{n}{surf\PYZus{}sim\PYZus{}param} \PY{o}{=} \PY{n}{tan\PYZus{}beta} \PY{o}{/} \PY{p}{(}\PY{n}{np}\PY{o}{.}\PY{n}{sqrt}\PY{p}{(}\PY{n}{H\PYZus{}0}\PY{o}{/}\PY{n}{L\PYZus{}0}\PY{p}{)}\PY{p}{)}
          \PY{c+c1}{\PYZsh{} deepwater height and length}
\end{Verbatim}


    \begin{Verbatim}[commandchars=\\\{\}]
{\color{incolor}In [{\color{incolor}116}]:} \PY{n}{Rs} \PY{o}{=} \PY{n+nb}{round}\PY{p}{(}\PY{p}{(}\PY{l+m+mf}{1.38} \PY{o}{*} \PY{n}{surf\PYZus{}sim\PYZus{}param} \PY{o}{*} \PY{n}{H\PYZus{}sig}\PY{p}{)}\PY{p}{,} \PY{l+m+mi}{2}\PY{p}{)}
          
          \PY{n+nb}{print}\PY{p}{(}\PY{l+s+s1}{\PYZsq{}}\PY{l+s+s1}{The wave run up }\PY{l+s+se}{\PYZbs{}n}\PY{l+s+s1}{ is }\PY{l+s+si}{\PYZob{}\PYZcb{}}\PY{l+s+s1}{ m}\PY{l+s+s1}{\PYZsq{}}\PY{o}{.}\PY{n}{format}\PY{p}{(}\PY{n}{Rs}\PY{p}{)}\PY{p}{)}
\end{Verbatim}


    \begin{Verbatim}[commandchars=\\\{\}]
The wave run up 
 is 8.62 m

    \end{Verbatim}

    \begin{enumerate}
\def\labelenumi{(\alph{enumi})}
\setcounter{enumi}{2}
\item
\end{enumerate}

    Wave setup is the mean increase in vertical water level due to wave
breaking. These values are usually pretty low, around 20\% of the
breaking wave height, but can reach up to approximately 1 m during
storms. From this problem, the wave setup is 0.648 to 0.810 m. The runup
is the uprush of water at the shore due to wave breaking and depends on
wave conditions, bototm slope, and 3d effects. Runup, by definition,
includes setup so it makes sense that it is significantly larger than
the setup value. So the vertical uprush of water, including setup (0.65
- 0.81 m) in total is 8.62 m.

    

    Problem 4

    Given: Tide data at station 8635750

Find: (a) tidal form (b) storm water level at mean sea level and mean
lower low water

Principles:

    \begin{enumerate}
\def\labelenumi{(\alph{enumi})}
\item
\end{enumerate}

    \begin{Verbatim}[commandchars=\\\{\}]
{\color{incolor}In [{\color{incolor}117}]:} \PY{n}{a\PYZus{}k1} \PY{o}{=} \PY{l+m+mf}{0.023}
          \PY{n}{a\PYZus{}o1} \PY{o}{=} \PY{l+m+mf}{0.019}
          \PY{n}{a\PYZus{}m2} \PY{o}{=} \PY{l+m+mf}{0.184}
          \PY{n}{a\PYZus{}s2} \PY{o}{=} \PY{l+m+mf}{0.028}
          
          \PY{n}{R} \PY{o}{=} \PY{n+nb}{round}\PY{p}{(}\PY{p}{(}\PY{n}{a\PYZus{}k1} \PY{o}{+} \PY{n}{a\PYZus{}o1}\PY{p}{)}\PY{o}{/}\PY{p}{(}\PY{n}{a\PYZus{}m2} \PY{o}{+} \PY{n}{a\PYZus{}s2}\PY{p}{)}\PY{p}{,} \PY{l+m+mi}{3}\PY{p}{)}
          
          \PY{n+nb}{print}\PY{p}{(}\PY{l+s+s1}{\PYZsq{}}\PY{l+s+s1}{The tidal form number is }\PY{l+s+si}{\PYZob{}\PYZcb{}}\PY{l+s+s1}{ m }\PY{l+s+se}{\PYZbs{}n}\PY{l+s+s1}{ this means the tide type is semidiurnal }\PY{l+s+se}{\PYZbs{}n}\PY{l+s+s1}{ since it is less than 0.25.}\PY{l+s+s1}{\PYZsq{}}\PY{o}{.}\PY{n}{format}\PY{p}{(}\PY{n}{R}\PY{p}{)}\PY{p}{)}
\end{Verbatim}


    \begin{Verbatim}[commandchars=\\\{\}]
The tidal form number is 0.198 m 
 this means the tide type is semidiurnal 
 since it is less than 0.25.

    \end{Verbatim}

    \begin{enumerate}
\def\labelenumi{(\alph{enumi})}
\setcounter{enumi}{1}
\item
\end{enumerate}


    % Add a bibliography block to the postdoc
    
    
    
    \end{document}
